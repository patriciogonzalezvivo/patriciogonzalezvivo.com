\documentclass[11pt,letterpaper]{article}
\usepackage[utf8]{inputenc}
\usepackage[margin=1in]{geometry}
\usepackage{graphicx}
\usepackage{float}
\usepackage{caption}
\usepackage{subcaption}
\usepackage{fancyhdr}
\usepackage{titlesec}
\usepackage{hyperref}
\usepackage{parskip}

% Set up page style
\pagestyle{fancy}
\fancyhf{}
\fancyhead[L]{Patricio Gonzalez Vivo}
\fancyhead[R]{Portfolio}
\fancyfoot[C]{\thepage}
\renewcommand{\headrulewidth}{0.4pt}

% Customize section titles
\titleformat{\section}
  {\Large\bfseries}
  {}
  {0em}
  {}
  [\titlerule]

\titleformat{\subsection}
  {\large\bfseries}
  {}
  {0em}
  {}

% Hyperlink styling
\hypersetup{
    colorlinks=true,
    linkcolor=black,
    urlcolor=blue,
}

% No paragraph indentation
\setlength{\parindent}{0pt}

\begin{document}

% Title page
\begin{titlepage}
    \centering
    \vspace*{2cm}
    
    {\Huge\bfseries Patricio Gonzalez Vivo}
    
    \vspace{1.5cm}
    
    {\Large Portfolio}
    
    \vspace{2cm}
    
    {\large Artist \& Educator}
    
    \vfill
    
    {\small \url{https://patriciogonzalezvivo.com}}
    
    \vspace{1cm}
    
\end{titlepage}

% Biography section
\section*{Biography}
\textbf{Patricio Gonzalez Vivo} (Buenos Aires, 1982) is a multidisciplinary artist whose practice spans traditional and digital media. Now based in the United States, he explores awareness, perception, and self-discovery through motifs such as celestial bodies, esoteric symbolism, temporal systems, and cartographic forms. His work weaves together the poetic and the procedural, revealing hidden structures behind the ways we measure and make sense of the world.

A defining aspect of Patricio’s practice is the creation of his own tools, libraries, and pedagogical resources, which he releases openly. For him, toolmaking is both an artistic method and a form of community stewardship, an act of sharing knowledge and expanding access to creative technologies. His Book of Shaders has shaped an entire generation of digital artists, serving as a cornerstone for workshops, courses, and derivative projects around the world. His tools -GlslViewer, glslCanvas and glslPipeline -and libraries such as Lygia, Vera, Berthe, and Hypatia are widely used as reliable instruments for artistic experimentation and production.

Patricio’s work has been exhibited and presented at EYEO, Resonate, GROW, FILE, Espacio Fundación Telefónica, BrightMoments, FlyingTokyo, DotDotDot and FASE. He has spoken about his practice at institutions including the MIT Media Lab, Carnegie Mellon University, Frank-Ratche Studio for Creative Inquiry and Politecnico di Milano. He has taught at Parsons School of Design (his alma mater, in the MFA Design \textbackslash{}& Technology program), ITP NYU, SFPC (School for Poetic Computation), and the Instituto Universitario Nacional de Arte in Argentina. His work has been featured in Wired, Gizmodo, The Atlantic, and Fast Company.

Collect | Contact

\newpage

\section*{Selected Works}

\subsection*{Astros}

\textit{2026 | Custom real-time software}

\begin{figure}[H]
    \centering
    \includegraphics[width=0.8\textwidth]{2026/astros/thumb.gif}
\end{figure}

Real-time astronomical visualization

\vspace{1cm}

\subsection*{Imaginary Portraits}

\textit{2025 | Mixed Media on Canvas}

\begin{figure}[H]
    \centering
    \includegraphics[width=0.8\textwidth]{2025/imaginary/thumb.jpg}
\end{figure}

Imaginary portrait studies

\begin{figure}[H]
    \centering
    \includegraphics[width=0.45\textwidth]{2025/imaginary/images/IMG_7615.jpeg}
    \hfill
    \includegraphics[width=0.45\textwidth]{2025/imaginary/images/IMG_7634.jpeg}
    \vspace{0.5cm}

    \includegraphics[width=0.45\textwidth]{2025/imaginary/images/IMG_7697.jpeg}
    \hfill
    \includegraphics[width=0.45\textwidth]{2025/imaginary/images/IMG_7713.jpeg}
\end{figure}

\vspace{1cm}

\subsection*{Hybrids Studies}

\textit{2025 | Oil on Canvas}

\begin{figure}[H]
    \centering
    \includegraphics[width=0.8\textwidth]{2025/hybrids/thumb.jpg}
\end{figure}

Hybrid portrait studies

\begin{figure}[H]
    \centering
    \includegraphics[width=0.45\textwidth]{2025/hybrids/images/IMG_7185.jpeg}
    \hfill
    \includegraphics[width=0.45\textwidth]{2025/hybrids/images/IMG_7330.jpeg}
    \vspace{0.5cm}

    \includegraphics[width=0.45\textwidth]{2025/hybrids/images/IMG_7375.jpeg}
    \hfill
    \includegraphics[width=0.45\textwidth]{2025/hybrids/images/IMG_7437.jpeg}
\end{figure}

\vspace{1cm}

\subsection*{BLINK}

\textit{2023 | Real-time Generative Art}

\begin{figure}[H]
    \centering
    \includegraphics[width=0.8\textwidth]{2023/blink/thumb.gif}
\end{figure}

Both memento mori and moment of delight, an object suspended between disappearance and wonder

\vspace{1cm}

\subsection*{Time Studies}

\textit{2022 | Video Art}

\begin{figure}[H]
    \centering
    \includegraphics[width=0.8\textwidth]{2022/time/thumb.gif}
\end{figure}

Video work exploring time, perception, and emotional states

\vspace{1cm}

\subsection*{Memory Studies}

\textit{2021 | Real-time Generative Art}

\begin{figure}[H]
    \centering
    \includegraphics[width=0.8\textwidth]{2021/memory/thumb.gif}
\end{figure}

Exploration of digital memory through sorting and scrambling algorithms

\vspace{1cm}

\subsection*{Flight Studies}

\textit{2021 | Red Oak, Ucycled Screen, custom real-time software | 52cm x 27cm x 5.5cm}

\begin{figure}[H]
    \centering
    \includegraphics[width=0.8\textwidth]{2021/fen/thumb.gif}
\end{figure}

Displayed on custom made digital frames, build from repurposed e-waste and reimagined as a minimalist object of contemplation.

\vspace{1cm}

\subsection*{Hogar}

\textit{2019 | Custom real-time software}

\begin{figure}[H]
    \centering
    \includegraphics[width=0.8\textwidth]{2019/hogar/thumb.gif}
\end{figure}

Real-time view of Earth from space

\vspace{1cm}

\subsection*{Estrellas}

\textit{2018 | Custom real-time software}

\begin{figure}[H]
    \centering
    \includegraphics[width=0.8\textwidth]{2018/estrellas/thumb.gif}
\end{figure}

Real-time star map and astronomical visualization

\vspace{1cm}

\subsection*{Luna}

\textit{2017 | Custom real-time software}

\begin{figure}[H]
    \centering
    \includegraphics[width=0.8\textwidth]{2017/luna/thumb.jpg}
\end{figure}

Living meditation on time and celestial rhythm

\vspace{1cm}

\subsection*{Skylines}

\textit{2014}

\begin{figure}[H]
    \centering
    \includegraphics[width=0.8\textwidth]{2014/skylines/thumb.png}
\end{figure}

\begin{figure}[H]
    \centering
    \includegraphics[width=0.45\textwidth]{2014/skylines/images/camera-obscura.jpg}
    \hfill
    \includegraphics[width=0.45\textwidth]{2014/skylines/images/electronic.jpg}
    \vspace{0.5cm}

    \includegraphics[width=0.45\textwidth]{2014/skylines/images/eye.jpg}
    \hfill
    \includegraphics[width=0.45\textwidth]{2014/skylines/images/hektor.jpg}
\end{figure}

\vspace{1cm}

\subsection*{Efectomariposa}

\textit{2011}

\begin{figure}[H]
    \centering
    \includegraphics[width=0.8\textwidth]{2011/efectomariposa/thumb.jpg}
\end{figure}

On Saturday June 4, 2011, after decades of inactivity, Puyehue Volcano ejected a plume of ash 10 kilometers high and 5 wide. This led to a natural disaster with strong environmental and financial impacts in the region. At the same time, the ash killing thousands of living beings ensures the fertility of the region for years to come

\begin{figure}[H]
    \centering
    \includegraphics[width=0.45\textwidth]{2011/efectomariposa/images/02.jpg}
    \hfill
    \includegraphics[width=0.45\textwidth]{2011/efectomariposa/images/03.jpg}
    \vspace{0.5cm}

    \includegraphics[width=0.45\textwidth]{2011/efectomariposa/images/04.jpg}
    \hfill
    \includegraphics[width=0.45\textwidth]{2011/efectomariposa/images/05.jpg}
\end{figure}

\vspace{1cm}

\subsection*{Communitas}

\textit{2010}

\begin{figure}[H]
    \centering
    \includegraphics[width=0.8\textwidth]{2010/communitas/thumb.jpg}
\end{figure}

\begin{quote}\textbackslash{}\textit{“When we forget ourselves, we are the universe..”\textbackslash{}} Hakui\end{quote}

\begin{figure}[H]
    \centering
    \includegraphics[width=0.45\textwidth]{2010/communitas/images/02.jpg}
    \hfill
    \includegraphics[width=0.45\textwidth]{2010/communitas/images/03.jpg}
    \vspace{0.5cm}

    \includegraphics[width=0.45\textwidth]{2010/communitas/images/04.jpg}
    \hfill
    \includegraphics[width=0.45\textwidth]{2010/communitas/images/05.jpg}
\end{figure}

\vspace{1cm}

\end{document}